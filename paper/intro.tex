\section{Introduction}

Initializers in range-based for loops~\cite{p0614} were introduced in C++20 to
solve issues with prvalues that might lead to undefined behavior.

This has led to a new common pattern while iterating:

\begin{lstlisting}
std::array vec {2, 4, 6};
for (int i=0; auto x : vec) {
  std::println("v[{}] = {}", i, x);
  ++i;
}
\end{lstlisting}
\begin{lstlisting}[style=terminal]
v[0] = 2
v[1] = 4
v[2] = 6
\end{lstlisting}

This allows safe traversal of an iterable, while having a handy way to access to
the integer index.

However, one might easily forget updating the index:

\begin{lstlisting}
std::array vec {2, 4, 6};
for (int i=0; auto x : vec) {
  std::println("v[{}] = {}", i, x);
  // Ouch! Forgot to update i
}
\end{lstlisting}
\begin{lstlisting}[style=terminal]
v[0] = 2
v[0] = 4
v[0] = 6
\end{lstlisting}

An obvious extension is to allow the increment to be part of the for statement.
P0614~\cite{p0614} already outlined as a future direction a range-based for with
increment as \cppkey{for (init; auto x : e; inc)}. However, to the best of my
knowledge this has never been proposed.
