\section{Proposed change}

\subsection{Before/After}

\begin{center}
\begin{tabular}{|p{.4\textwidth}|p{.4\textwidth}|}
\hline
\textbf{Before} & \textbf{After}\\
\hline
\begin{lstlisting}
for (int i = 0; auto x : e) {
  f(x,i);
  ++i;
}
\end{lstlisting}
&
\begin{lstlisting}
for (int i = 0; auto x : e; ++i) {
  f(x,i);
}
\end{lstlisting}
\\
\hline
\end{tabular}
\end{center}

\subsection{Discussion}

\subsubsection{Current use}

The use of initializers in for-statements is encouraged by some coding guidelines
(e.g.
\href{https://isocpp.github.io/CppCoreGuidelines/CppCoreGuidelines#es6-declare-names-in-for-statement-initializers-and-conditions-to-limit-scope}{CppCoreGuidelines
ES.6}). However, the fact that the initialized variable needs to be updated in
loop body (and might be forgotten) might lead to logical errors.

This current state has also a teachability impact as the programmer has to
remember to add the update of the initialized variable at the end of the loop
body.

\subsubsection{Expected meaning}

It might be considered that the proposed meaning is surprising as a possible
interpretation is that in a traditional for loop and a range-based for loop
have similarities:

\begin{lstlisting}
for (int i=0; i<max; ++i) {
  //...
}

for (auto x : v) {
  //...
}
\end{lstlisting}

Consequently the expression \cppid{x : v} may be seen as a termination
condition, just like the expression \cppid{i < max}.

However, consider the case of the proposed new syntax.
\begin{lstlisting}
for (int i=0; x : v; ++i) {
  //...
}
\end{lstlisting} 

In this case, the programmer might see that the apparent loop variable is
\cppid{i} and that it is used in the loop condition.

However, a different view is that we can read \cppkey{for auto x : v} as
\emph{``for each value x in the range v''}. This is more consistent with the
name \emph{range-based for loop}.

The fact that we have an unrelated variable \cppid{i} was introduced in C++20. So,
since then we have the syntax for \emph{range-based for loop with initializers}:

\begin{lstlisting}
for(int i=0; auto x: v) {
  //...
}
\end{lstlisting}

Thus, in fact, we have \textbf{two} loop variables introduced: \cppid{i} and
\cppid{x}. The latter (\cppid{x}) is implicitly updated in the for statement,
while the former (\cppid{i}), needs to be updated in the loop body and may be
easily forgotten.
 
\subsubsection{Considering the for-range declaration different from the
condition}

An alternate design might consider the \emph{for-declaration} different from the
condition.

I am \textbf{not proposing} all 
the following possibilities:

\begin{lstlisting}
// Traditional for loop
for (int i=0; i<max; ++i) {
  //...
}

// Range-based for loop
for (x : v) {
  //...
}

// Range-based for loop with initializer
for (int i=0; x : v) {
  //...
  ++i;
}

// Range-based for loop with initializer and condition
for (int i=0; x : v; i<10) {
  //...
  ++i
}

// Range-based for loop with initializer, condition and expression
for (int i=0; x : v; i<10; ++i) {
  //...
}

// Range-based for loop with initializer, condition and without expression
for (int i=0; x : v; ; ++i) {
  //...
}
\end{lstlisting}

This would be more complicated from what I currently propose in this paper.

\subsubsection{Conflict with multi-range for loops}

A conflicting proposal would be to allow multiple ranges in the same for-loop as
outlined by P0026~\cite{p0026}. 

\begin{lstlisting}
for (int i = 0; auto x : e1; auto y : e2) {
  f(x,y);
}
\end{lstlisting}

However, such a proposal was presented in 2016
and since then there has been no activity in such direction.

Nevertheless, I am not preventing this approach if it is retaken as a
\cppid{for-range declaration} is different from an expression.

\subsubsection{Library solutions}

Although a library solution has not been explored, one might envision such a
solution based on some kind of iterator:

\begin{lstlisting}
for (auto i : indexed{e}) {
  std::println("v[{}] = {}", i.index(), i.value());
}
\end{lstlisting}

Or even in combination with structured bindings:

\begin{lstlisting}
for (auto [i, x] : indexed{e}) {
  std::println("v[{}] = {}", i, x);
}
\end{lstlisting}

However, those solutions assume a specific type for the index (\cppkey{int},
\cppkey{std::size\_t}, \ldots). Besides they do not allow for cases where the
updated variable(s) might get an update different from plain increment.

\subsubsection{Using \cppid{views::enumerate}}

One specific solution would be to use \cppid{std::views::enumerate}.

\begin{lstlisting}
std::array v{2.5, 3.0, 4.5, 5.0, 5.5};
for (auto [i, x] : std::views::enumerate(v)) {
  std::println("v[{}] = {}", i, x);
}
\end{lstlisting}

However, this solution has two drawbacks:
\begin{itemize}

\item 
It is not possible to specify the specific type of the integer index.

\item 
The initial value of the index is fixed to start at \cppid{0}.

\end{itemize}

\subsubsection{Current approach}

Current approach for the example covered in this paper could be:
\begin{lstlisting}
std::array vec {1, 2, 3, 4, 5, 6};
for (int i=0; auto x : vec) {
  std::println("v[{}] = {}", i++, x);
}
\end{lstlisting}

However, this presents a maintenance burden. Consider now, that we decide to
modifiy the loop to print only even values:

\begin{lstlisting}
std::array vec {1, 2, 3, 4, 5, 6};
for (int i=0; auto x : vec) {
  if (is_even(x)) {
    std::println("v[{}] = {}", i++, x);
  }
}
\end{lstlisting}

That would incorrectly print indices of even values as the index is only
incremented if \cppid{is\_even(x)} is \cppkey{true}.

However, consider both loops with the proposed solution:
\begin{lstlisting}
std::array vec {1, 2, 3, 4, 5, 6};
for (int i=0; auto x : vec, ++i) {
  std::println("v[{}] = {}", i, x);
}
for (int i=0; auto x : vec, ++i) {
  if (is_even(x)) {
    std::println("v[{}] = {}", i, x);
  }
}
\end{lstlisting}

