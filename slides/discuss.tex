\section{Considerations}

\subsection{Meaning}

\begin{frame}[t,fragile]{Expected meaning}
\begin{itemize}
  \item \textbad{Drawback?}: The apparent loop variable \cppid{i} is not used in the loop
        condition.
\begin{lstlisting}
for (int i=0; auto const & x : vec; ++i) { /*...*/ }
\end{lstlisting}

  \mode<presentation>{\vfill\pause}
  \item \textgood{Answer}:
        The fact that variable \cppid{i} is unrelated with range-for expression
        was introduced in C++20.
\begin{lstlisting}
for (int i=0; auto const & x : vec) { /*...*/ }
\end{lstlisting}
    \begin{itemize}
      \item Two variables: 
            \cppid{x} is implicitly updated in the loop statement,
            but \cppid{i} needs to be explicitly updated in the loop body.
    \end{itemize}


  \mode<presentation>{\vfill\pause}
  \item \textgood{Answer}:
        We can read \emph{starting from \cppid{i=0}, for every value \cppid{x}
        in \cppid{vec}, update \cppid{i} at the end of each iteration}.
        
\end{itemize}
\end{frame}

\subsection{Library solution}

\begin{frame}[t,fragile]{std::views::enumerate}
\begin{itemize}
  \item Some situations are solved by \cppid{std::views::enumerate}:
\begin{lstlisting}
for (auto [i, x] : std::views::enumerate(v)) {
  std::println("v[{}] = {}", i, x);
}
\end{lstlisting}

  \mode<presentation>{\vfill\pause}
  \item \textbad{Drawbacks}:
    \begin{itemize}
      \item The initial value for \cppid{i} cannot be specified.
      \item The type for \cppid{i} cannot be specified.
      \item The update expression is fixed to \cppid{++i}.
      \item The type of \cppid{x} is fixed to l-value reference.
    \end{itemize}

\end{itemize}
\end{frame}

\begin{frame}[t,fragile]{Initial index value}
\begin{itemize}
  \item The index value of \cppid{std::ranges::enumerate\_view} starts in
        \cppid{0} and is incremented in each iteration.

  \vfill\pause
  \item \textbad{Drawback}:
    \begin{itemize}
      \item Cannot start in a different value.
      \item This can be solved by simple arithmetic.
    \end{itemize}

\begin{lstlisting}
for (int i=42; auto x : vec; ++i) {
  other[i] = x;
}

for (auto [i,x] : std::views::enumerate(v)) {
  other[i+42] = x;
}
\end{lstlisting}
\end{itemize}
\end{frame}

\begin{frame}[t,fragile]{Type of introduced variable}
\begin{itemize}
  \item The type of the introduced index variable is from
     \cppid{std::ranges::enumerate\_view<V>::iterator::difference\_type}.
    \begin{itemize}
      \item Typically \cppid{std::ptrdiff\_t}.
    \end{itemize}

  \vfill\pause
  \item \textbad{Drawbacks}:
    \begin{itemize}
      \item Forces to use a signed type.
      \item Does not allow user defined types.
    \end{itemize}

\begin{columns}[T]

\column{.4\textwidth}
\begin{lstlisting}
for (std::size_t i = 0; x : vec; ++i) {
  other[i] = f(x);
}

for (my_index i{}; x : vec; ++i) {
  table.register_value(i, x);
}
\end{lstlisting}

\pause
\column{.4\textwidth}
\begin{lstlisting}
for (auto [i, x] : std::views::enumerate(v)) {
  other[static_cast<std::size_t>(i)] = f(x);
}

for (auto [i, x] : std::views::enumerate(v)) {
  k = // Is there a mapping from integers?
  table.register_value(k,x);
}
\end{lstlisting}

\end{columns}

\end{itemize}
\end{frame}

\begin{frame}[t,fragile]{Update expression}
\begin{itemize}
  \item The index value is updated by invoking \cppid{++i} where \cppid{i} is
        from a distance type (as \cppid{ptrdiff\_t}).

  \vfill\pause
  \item \textbad{Drawbacks}:
    \begin{itemize}
      \item Does not allow a different step value.
      \item Does not allow a different update expression.
    \end{itemize}

\begin{lstlisting}
for (std::size_t i = 100; x : vec; i += 2) {
  table[i] = f(x);
}

for (std::size_t i = 0; x : vec; i = (i + 1) % 32) {
  table[i] = f(x);
}

for (my_index i{}; x : vec; i.advance()) {
  table.register_value(i, x);
} 
\end{lstlisting}

\end{itemize}
\end{frame}

\begin{frame}[t,fragile]{Non-integer indexing}
\begin{lstlisting}
enum class flag { none=0, open=1<<0, close=1<<1, read=1<<2, write=1<<3 };

flag next(flag x); // Get next flag option

std::array names { "<none>", "<open>", "<close>", "<read>", "<write>"};

void fun() {
  for (auto f = flag::none; auto n : names; f = next(f)) {
    std::println("{}", n);
    do_something(f);
  }
}
\end{lstlisting}
\end{frame}

\begin{frame}[t,fragile]{Type of iterated value}
\begin{itemize}
  \item The iterated value (second element of the tuple) is an l-value reference
        to the values in the iterated range.

  \item \textbad{Drawbacks}:
    \begin{itemize}
      \item Cannot iterate by value.
      \item Cannot iterate by const l-value reference.
      \item Cannot iterate by r-value reference.
      \item Cannot iterate by forwarding reference.
    \end{itemize}

  \item Iteration by const l-value reference can be done with
        \cppid{std::as\_const()}
\begin{lstlisting}
for (auto [i, x] : std::views::enumerate(std::as_const(v))) {
  f(i,x);
}
\end{lstlisting}

  \item Others are not easy to achieve.

\end{itemize}
\end{frame}
